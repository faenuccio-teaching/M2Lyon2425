\documentclass[12pt,a4paper]{article}

\usepackage[latin1]{inputenc}
%\usepackage[utf8]{inputenc}
\usepackage{babel}
\usepackage{graphicx}
\usepackage[margin=1.6cm]{geometry}
\usepackage{amsfonts}
\usepackage{amsmath,amsthm}
\usepackage{hyperref}
\usepackage{float}
\usepackage{subfig}
\usepackage{mathrsfs}

\newtheorem*{theorem}{Theorem}
\newtheorem*{proposition}{Proposizione}
\newtheorem*{corollary}{Corollario}
\newtheorem*{lemma}{Lemma}
\newtheorem*{definition}{Definizione}

\title{Project proposal}
\author{Andrea Iacco}
\begin{document}
\maketitle

Good evening, I am very sorry I am this late in submitting my idea for the project, but since it involved topology, I wanted to wait for the first class about topological spaces, which was two days ago.
I mistakenly assumed that what you were asking us to do here was to already give an outline of the Lean tactics that we would use for the project. After talking with Prof. Morel last Wednesday I learnt that 
was not the case. So here I come with my idea, which, in fact, would be to formalize a famous result from topology.

I would like to take on part of the proof of Van Kampen's theorem. I would have the following statement:
\begin{theorem}[Van Kampen]
Let $X$ be a topological space, and let $A$,$B \subseteq X$ two open subsets of $X$ such that:
1) $A$, $B$, and $A \cap B$ are path connected
2) $A \cup B = X$
Let $x_0 \in A \cap B$, $f: A \to X$, $g: B \to X$ the inclusion maps. Let $P_A = \pi_1 (A,x_0)$, $P_B = \pi_1 (B,x_0)$, $P_{AB} = \pi_1 (A \cap B,x_0)$, $P = \pi_1 (X,x_0)$ the fundamental groups of $A$, $B$, $A \cap B$, $X$ centered in $x_0$.
Let $f_\ast$, $g_\ast$ be the images of $f$ and $g$ under the functor $\pi_1$: Pointed Top Spaces $\to$ Groups. Let's call $\star$ the junction between two paths $\alpha$ and $\beta$ on $X$ such that $\alpha(1) = \beta(0)$. 
So $(\alpha \star \beta): [0,1] \to X$, $(\alpha \star \beta)(t) = \begin{cases}
\alpha(2t) & \text{if } t \leq 1/2\\ 
\beta(2t-1) & \text{if } t \geq 1/2\\
\end{cases}$. For each path $\delta$ on $X$ let's call $inv(\delta)$ the inverse path,
meaning $inv(\delta): [0,1] \to X$, $inv(\delta)(t) = \delta(1-t)$. 
Let $P_A \ast P_B$ be the free product of $P_A$ and $P_B$ (so the coproduct in the category of Groups), and $\ast$ be the free product operation in this group. Then the function $\phi: P_A \ast P_B \to P$ that sends 
$\alpha_1 \ast \beta_1 \ast \cdots \ast \alpha_n \ast \beta_n$ to $f_\ast (\alpha_1) \star g_\ast (\beta_1) \star \cdots \star f_\ast (\alpha_n) \star g_\ast (\beta_n)$ (which is the one deriving by the universal property of 
the coproduct in the Group category) is surjective.
Also, if $i_1: A \cap B \to A$ and $i_2 : A \cap B \to B$ are the inclusion maps, $j_1: P_A \to P_A \ast P_B$ and $j_2: P_B \to P_A \ast P_B$ are the canonical maps for the coproduct, 
and $i_{1_\ast}$, $i_{2_\ast}$ are the images of $i_1$ and $i_2$ under the functor $\pi_1$, then we have $\{j_1(i_{1_\ast}(\alpha)) \star j_2(i_{2_\ast}(inv(\alpha))),\, \alpha \in P_{AB}\} \subseteq \mathrm{Ker}(\phi)$.
\end{theorem}

Actually, this last inclusion is an equality, but proving the other inclusion is already hard enough in regular math classes, so I don't think it would be a good idea to also formalize that part of the Van Kampen theorem.
\ \\

The proof I have of this theorem starts from a preliminary lemma, with the following statement: 
\begin{lemma}
Let $X$ be a topological space, and $\mathcal{A}$ an open covering of $X$. Let $\alpha: [0,1] \to X$ be a path on $X$. Then there exists a positive integer $n$ such that, for all $i \in \{0,\ldots,n-1\}$, there exists a 
$B_i \in \mathcal{A}$ such that $\alpha ([\frac{i}{n}, \frac{i+1}{n}]) \subseteq B_i$.
\end{lemma} 

The proof of this preliminary lemma is as follows:

\begin{proof}
For each $B \in \mathcal{A}$, we have that $\alpha^{-1}(B)$ is an open set in $[0,1]$, thus given by disjoint unions of open real intervals, which are in turn intersected with $[0,1]$.\\ 
Consider then the set $\mathcal{B} = \left\{I \subseteq [0,1] \text{ such that I is open, and there exists } B \in \mathcal{A} \text{ such that } I \text{ is a}\right.$ connected component of $\left.\alpha^{-1}(B)\right\}$. Since $\mathcal{A}$ is an open covering of $X$
and the union of all sets in $\mathcal{B}$ is the same as the union of all elements of $\{\alpha^{-1}(B) \text{ such that } B \in \mathcal{A}\}$, then $\mathcal{B}$ is an open covering of $[0,1]$. Since $[0,1]$ is compact, we may then extract
a finite subcovering from $\mathcal{B}$, let's call it $\mathcal{C}$. Each element of $\mathcal{C}$ is an open connected subset of $[0,1]$, and is also a connected component of some $\alpha^{-1}(B)$, for $B \in \mathcal{A}$, by definition.
We can then say that $\forall I \in \mathcal{C},\, \exists B \in \mathcal{A} \text{ such that }\alpha(I) \subseteq B$. Now consider the biggest interval in $\mathcal{C}$ that contains 0, let's call it $I_0$. If that interval is $[0,1]$ then we are done,
setting $n = 1$. If that interval is $[0,x_1)$ then consider the intervals in $\mathcal{C}\setminus \{I_0\}$ that contain $x_1$. These intervals will be of the form $(x_1-\lambda_1,1]$ or $(x_1-\lambda_1, x_1+\rho_1)$, with $\lambda_1 > 0$, $\rho_1 > 0$. Select the one with the biggest right bound, and call it $I_1$. In both cases we have that $[0,x1) \cap I_1 \neq \emptyset$. If applicable, call $x_2 = x_1+\rho_1$. Continue this process until we find an interval $I_k$ that contains 1.\\ 
This is possible because $\mathcal{C}$ contains finitely many intervals, and at each step we progressively remove them one by one. We now have open intervals $I_0,\ldots,I_k$ such that $I_{i-1} \cap I_{i} \neq \emptyset$ for all $i \in \{1,\ldots,k\}$. Set $n$ such that $\frac{1}{n}$ is smaller than all intervals $I_{i-1} \cap I_i$, for all $i \in \{1,\ldots,k\}$ (again, a finite number of intervals). Then each interval $[\frac{j}{n}, \frac{j+1}{n}]$ is a subset of a specific $I_i$, 
for all $j \in \{0,\ldots,n-1\}$, and this can be proved by induction on $j$.

As base case, consider $j=0$.
The interval $[0, \frac{1}{n}]$ is included in $I_0$, as $I_0$ is bigger than $I_0 \cap I_1$, which means $x1 = |x1 - 0| > \frac{1}{n} = |1/n - 0|$, and thus $[0,\frac{1}{n}] \subseteq [0,x1) = I_0$.\\

Now suppose that $[\frac{j-1}{n}, \frac{j}{n}]$ is included into some $I_i$. 

Suppose $i < k$. If $[\frac{j-1}{n},\frac{j}{n}] \cap (I_i \cap I_{i+1}) = \emptyset$, then the set $I_i \cap [\frac{j}{n}, 1]$ strictly contains $(I_i \cap I_{i+1})$, thus if $g = \sup (I_i \cap I_{i+1})$, $|g-\frac{j}{n}|> \frac{1}{n} = |\frac{j+1}{n} - \frac{j}{n}|$. As such,
$[\frac{j}{n}, \frac{j+1}{n}] \subseteq I_i$. If $[\frac{j-1}{n}, \frac{j}{n}] \cap (I_i \cap I_{i+1}) \neq \emptyset$, then $\frac{j}{n} \in (I_i \cap I_{i+1})$. If $i+1 = k$, then $[\frac{j}{n}, \frac{j+1}{n}]$ must be a subset of $I_k$, as $I_k$ is the interval containing 1, and $\frac{j+1}{n} \leq 1$. If $i+1 < k$, we have that $\frac{j}{n}$ cannot be contained into $(I_{i+1} \cap I_{i+2})$. This is because $\frac{j}{n} < x_{i+1}$, and if it were part of $I_{i+2}$, then $x_{i+1}$ would also be part of that interval. This contradicts our choice for the interval $I_{i+1}$ as the one containing $x_{i+1}$ that had the greatest right bound ($I_{i+2}$ has an even greater right bound, as it contains $x_{i+2}$ that wasn't contained into $I_{i+1}$ by definition).
Thus $[\frac{j}{n}, 1] \cap I_{i+1}$ contains $(I_{i+1} \cap I_{i+2})$, and following the same reasoning applied before, we have $[\frac{j}{n}, \frac{j+1}{n}] \subseteq I_{i+1}$. \\

If $i = k$, then $I_i$ contains 1. If $j = k$ there is nothing to prove, if $j < k$ then $\frac{j+1}{n} \leq 1$, so $[\frac{j}{n}, \frac{j+1}{n}] \subseteq I_k$.\\
So now we successfully partitioned $[0,1]$ into intervals $[\frac{j}{n}, \frac{j+1}{n}]$ such that each of them is contained into one respective $I_i$. But each $I_i$ is such that $\alpha (I_i)$ is contained into one of the open sets of $\mathcal{A}$. For all $j$, consider $B_j$ as the open set in $\mathcal{A}$ that contains $\alpha(I_i)$, where $[\frac{j}{n}, \frac{j+1}{n}] \subseteq I_i$.  We have then found our partition, such that $\alpha ([\frac{j}{n}, \frac{j+1}{n}]) \subseteq B_j$, as we wanted.
\end{proof}

Given this preliminary lemma, the proof of the Van Kampen theorem as stated above is as follows:

\begin{proof}
Let $\alpha$ be a loop in $X$ with base point $x_0$ ($\alpha\in P$). Then, since $A$ and $B$ are open, by the lemma there exists a positive integer $n$ such that, for all $i \in \{0,\ldots,n\}$, $\alpha([\frac{i}{n}, \frac{i+1}{n}]) \subseteq A$ or $\alpha([\frac{i}{n}, \frac{i+1}{n}]) \subseteq B$. Let's call, for each $i$, $x_i = \alpha(\frac{i}{n})$. $x_0$ found this way is exactly the $x_0$ appearing in the hypothesis, as it is the base point of $\alpha$, and it is also equal to $x_n$.\\
Let's define, for each $0\leq i < n$, a path from $x_i$ to $x_{i+1}$. That is $\alpha_{i}: [0,1] \to X$ such that $\alpha_i(t) = \alpha (\frac{i+t}{n})$. Now, consider one $x_i$. If $x_i \in A \cap B$, since $A \cap B$ is path connected, we can define a path $\beta_i: [0,1] \to A \cap B$ from $x_i$ to $x_0$ (we define the paths $\beta_0$ and $\beta_n$ as the trivial path). If $x_i \in (A \setminus B)$, since $A$ is also path connected, we may define a path $\beta_i: [0,1] \to A$ from $x_i$ to $x_0$. If $x_i \in (B \setminus A)$, since $B$ is also path connected, we may define a path $\beta_i: [0,1] \to B$ from $x_i$ to $x_0$.\\

Let's now define paths $\gamma_1,\ldots,\gamma_n$ in the following way. For each $i \in \{0,\ldots,n-1\}$, $\gamma_{i+1} = inv(\beta_{i}) \star \alpha_i \star \beta_{i+1}$ (remember $\beta_0$ and $\beta_n$ are the trivial paths). Each of these paths is a loop with base point $x_0$, and by construction each of them has its image completely in $A$ or completely in $B$. Furthermore, \begin{flalign*}
\gamma_1 \star \gamma_2 \star \cdots \star \gamma_n & = inv(\beta_0) \star \alpha_0 \star \beta_1 \star inv(\beta_1) \star \alpha_1 \star \beta_2 \star \cdots \star \beta_{n-1} \star inv(\beta_{n-1}) \star \alpha_{n-1} \star \beta_n\\
\ & = \alpha_0 \star \alpha_1 \star \cdots \star \alpha_{n-1} = \alpha\\
\end{flalign*}
by construction. As the $\gamma_i$ paths are either entirely in $A$ or entirely in $B$ each, each of them is either an element of $P_A$ or of $P_B$, while $\alpha$ is in $P$. We have thus shown that each $\alpha \in P$ can be written as the image under $\phi$ of  $\gamma_1 \ast \gamma_2 \ast \cdots \ast \gamma_{n-1}$, and can conclude that $\phi$ is surjective, as we wanted.\\

As for the second part of the theorem, we start by noting that $f \circ i_1 = g \circ i_2$, as they are both the inclusion map from $A \cap B$ to $X$. Since $\pi_1$ is a functor, then, $f_\ast \circ i_{1_\ast} = g_\ast \circ i_{2_\ast}$. Now, the universal property of coproducts also tells us that $f_\ast = \phi \circ j_1$ and $g_\ast = \phi \circ j_2$. Let then $\alpha \in P_{AB}$. We have $f_\ast \circ i_{1_\ast} (\alpha) = g_\ast \circ i_{2_\ast} (\alpha)$, which means $\phi(j_1(i_{1_\ast} (\alpha))) = \phi(j_2(i_{2_\ast} (\alpha)))$, which in turn implies $(j_1(i_{1_\ast}(\alpha)) \ast inv(j_2(i_{2_\ast}(\alpha))))\in\text{ Ker}(\phi)$. Since, given $\delta\in P$, $inv(\delta)$ is actually the inverse of $\delta$ in the group $P$, with binary operation $\star$, and since $j_2$ and $i_{2_\ast}$ are group homomorphisms, we may conclude $(j_1(i_{1_\ast}(\alpha)) \ast j_2(i_{2_\ast}(inv(\alpha))))\in\text{ Ker}(\phi)$, as we wanted.
\end{proof}

I tried to expand these proofs with as many details as I could, since these would also probably be required steps to do when formalizing. I am still missing a big portion of knowledge on which tactics I could
be using on Lean to actually prove all this stuff, and in topology class when we went over these proofs a lot was actually left as exercise or skipped because it was intuitive (which is a common thing to do in topology).\\
\ \\
I am therefore asking you if some doable project could be extracted from these theorems and their proofs that I have just written above, maybe if my initial idea was too ambitious I can prove only part of the things stated
above. If there are things that I introduced here that would be too hard to use on Lean (I'm thinking about the definition of fundamental group, and its usage as a functor, or the characterizations of open sets in $[0,1]$)
please let me know, and I will think of something more feasible.\\

Thank you for your attention.

Kind regards,

Andrea Iacco
\end{document}
